% below is for coding
\usepackage{listings}
\usepackage{xcolor}
\definecolor{codegreen}{rgb}{0,0.6,0}
\definecolor{codegray}{rgb}{0.5,0.5,0.5}
\definecolor{codepurple}{rgb}{0.58,0,0.82}
\definecolor{codeblue}{rgb}{0.30,0.30,0.82}
\definecolor{backcolour}{rgb}{0.95,0.95,0.95}
%Define the listing package
\usepackage{listings} %code highlighter
\usepackage{color} %use color
\definecolor{mygreen}{rgb}{0,0.6,0}
\definecolor{mygray}{rgb}{0.5,0.5,0.5}
\definecolor{mymauve}{rgb}{0.58,0,0.82}

%Customize a bit the look
\lstset{ %
	backgroundcolor=\color{white}, % choose the background color; you must add \usepackage{color} or \usepackage{xcolor}
	basicstyle=\footnotesize, % the size of the fonts that are used for the code
	breakatwhitespace=false, % sets if automatic breaks should only happen at whitespace
	breaklines=true, % sets automatic line breaking
	captionpos=b, % sets the caption-position to bottom
	commentstyle=\color{mygreen}, % comment style
	deletekeywords={...}, % if you want to delete keywords from the given language
        escapeinside={<@}{@>}, % if you want to add LaTeX within your code
	extendedchars=true, % lets you use non-ASCII characters; for 8-bits encodings only, does not work with UTF-8
	frame=single, % adds a frame around the code
	keepspaces=true, % keeps spaces in text, useful for keeping indentation of code (possibly needs columns=flexible)
	keywordstyle=\color{blue}, % keyword style
	% language=Octave, % the language of the code
	morekeywords={*,...}, % if you want to add more keywords to the set
	numbers=left, % where to put the line-numbers; possible values are (none, left, right)
	numbersep=5pt, % how far the line-numbers are from the code
	numberstyle=\tiny\color{mygray}, % the style that is used for the line-numbers
	rulecolor=\color{black}, % if not set, the frame-color may be changed on line-breaks within not-black text (e.g. comments (green here))
	showspaces=false, % show spaces everywhere adding particular underscores; it overrides 'showstringspaces'
	showstringspaces=false, % underline spaces within strings only
	showtabs=false, % show tabs within strings adding particular underscores
	stepnumber=1, % the step between two line-numbers. If it's 1, each line will be numbered
	stringstyle=\color{mymauve}, % string literal style
	tabsize=2, % sets default tabsize to 2 spaces
	title=\lstname % show the filename of files included with \lstinputlisting; also try caption instead of title
}
%END of listing package%

\definecolor{darkgray}{rgb}{.4,.4,.4}
\definecolor{purple}{rgb}{0.65, 0.12, 0.82}

%define Javascript language
\lstdefinelanguage{JavaScript}{
	keywords={typeof, new, true, false, catch, function, return, null, catch, switch, var, if, in, while, do, else, case, break},
	keywordstyle=\color{blue}\bfseries,
	ndkeywords={class, export, boolean, throw, implements, import, this},
	ndkeywordstyle=\color{darkgray}\bfseries,
	identifierstyle=\color{black},
	sensitive=false,
	comment=[l]{//},
	morecomment=[s]{/*}{*/},
	commentstyle=\color{purple}\ttfamily,
	stringstyle=\color{red}\ttfamily,
	morestring=[b]',
	morestring=[b]"
}

%define empty language
\lstdefinelanguage{empty}{
}
\lstdefinelanguage{ThesisHaskell}{
}
\lstdefinestyle{thesisStyle}{
    language=Haskell,
    % basicstyle=\rmfamily,
    commentstyle=\it,
    % stringstyle=\mdseries\rmfamily,
    showspaces=false,
    showstringspaces=false,
    % keywordstyle=\bfseries\rmfamily,
    deletekeywords={exp,String,Int,Bool,not,max,min,zero},
    morekeywords={forall},
    % identifierstyle=\it\rmfamily,
    morecomment=[l]{--},
    morecomment=[s]{\{-}{-\}},
    literate={+}{{$+$}}1 {/}{{$/$}}1 {*}{{$*$}}1 {=}{{$=$}}1 {>}{{$>$}}1 {<}{{$<$}}1 
           {\\}{{$\lambda$}}1
           {\\\\}{{\char`\\\char`\\}}1 {|}{{$\mid$}}1
           {->}{{$\rightarrow$}}2 
           {\\n}{{\textbackslash n}}2 
           {>=}{{$\geq$}}2 {<-}{{$\leftarrow$}}2
           {<=}{{$\leq$}}2 {=>}{{$\Rightarrow$}}2
           {>>}{{>>}}2 {>>=}{{>>=}}2
           {\\\$}{{{\$}}}2
  ,
  columns=flexible,
  emphstyle={\bf},
  mathescape=false,
  % emphstyle={[5]\sf\rmfamily},
  emph= {[5]
         zero, transfer, ifWithin,
         acc, rObs, bObs,letc
	    },
}
\lstnewenvironment{hscode}
   {\lstset{style=thesisStyle,frame=tlrb}}
   {}
   
\lstnewenvironment{hscodesmall}
   {\lstset{basicstyle=\tiny\ttfamily,style=thesisStyle,frame=tb}}
   {}

\lstset{
	language=JavaScript,
	extendedchars=true,
	basicstyle=\footnotesize\ttfamily,
	showstringspaces=false,
	showspaces=false,
	numbers=left,
	numberstyle=\footnotesize,
	numbersep=9pt,
	tabsize=2,
	breaklines=true,
	showtabs=false,
	captionpos=b
}
\lstdefinestyle{mystyle}{
	backgroundcolor=\color{backcolour},   
	commentstyle=\color{codegreen},
	keywordstyle=\color{codeblue},
	numberstyle=\tiny\color{codegray},
	stringstyle=\color{codepurple},
	basicstyle=\ttfamily\footnotesize,
	breakatwhitespace=false,         
	breaklines=true,                 
	captionpos=b,                    
	keepspaces=true,                 
	numbers=left,                    
	numbersep=5pt,                  
	showspaces=false,                
	showstringspaces=false,
	showtabs=false,                  
	tabsize=2
}
\lstset{style=mystyle}